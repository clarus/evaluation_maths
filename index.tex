\documentclass[a4paper]{article}
\usepackage[utf8]{inputenc}
\usepackage[T1]{fontenc}
\usepackage{lmodern}
\usepackage[francais]{babel}
\usepackage{fullpage}
\usepackage[hidelinks]{hyperref}
\usepackage{graphicx}
\usepackage{amsmath,amsfonts,amssymb}
\usepackage{tikz}
\usepackage{eurosym}

\begin{document}
  \title{Évaluation de Mathématiques - Bac Pro}
  \author{Module 1~: prérequis}
  \date{
    Nom~: .......................................\\
    \vspace{0.2cm}
    Date~: .......................................}
  \maketitle
  
  \section*{Puissances}
  \paragraph{Exercice 1}
  Écrire sous la forme d'une puissance de dix (en écrivant les étapes de calcul).
  \begin{align*}
    \frac{10^{-4}}{10^2} &\ = \ \dots\\
    (10^{-2})^2 &\ = \ \dots\\
    (10^0)^1 &\ = \ \dots\\
    10^2 \times 10^{-1} &\ = \ \dots\\
    \frac{10^{-6}}{10^{-6}} &\ = \ \dots
  \end{align*}

  \paragraph{Exercice 2}
  Transformer l'écriture en une seule puissance (en écrivant les étapes de calcul).
  \begin{align*}
    7^7 \times 7^3 &\ =\ \dots\\
    (-5)^3 \times (-5)^2 \times (-5) &\ =\ \dots\\
    (6^3)^4 &\ =\ \dots\\
    \frac{8^9}{8^6} &\ =\ \dots\\
    \frac{(-3)^2}{(-3)^4} &\ =\ \dots
  \end{align*}

  \section*{Expressions algébriques}
  \paragraph{Exercice 1}
  Soit $A = a - b + c$. En écrivant les étapes de calcul, calculer $A$ pour~:
  \[
    a = 2~; b = 3~; c = 1
  \]
  \vspace{2cm}

  \paragraph{Exercice 2}
  Soir $B = 2 (a + b) - 3 (a - b)$. En écrivant les étapes de calcul, calculer $B$ pour~:
  \[
    a = -3~; b = 2
  \]
  \vspace{2cm}

  \section*{Pourcentages}
  \paragraph{Exercice 1}
  On appelle $K$ le coefficient d'augmentation ou de réduction. Calculer $K$ pour~:
  \begin{itemize}
    \item une réduction de $30~\%$~:
      \[
        K \ =\ \dots
      \]
    \item une augmentation de $20~\%$~:
      \[
        K \ =\ \dots
      \]
  \end{itemize}
  \emph{Rappels~:}
  \[
    \mathrm{prix\ initial} \times K = \mathrm{prix\ final}
  \]
  \[
    \frac{\mathrm{prix\ final}}{K} = \mathrm{prix\ initial}
  \]

  \paragraph{Exercice 2}
  Un pantalon coûte 32~\euro\ après avoir subi une réduction de $30~\%$. En utilisant $K$ et en écrivant les étapes de calcul, calculer le prix initial du pantalon.
  \vspace{3cm}

  \section*{Développement et factorisation}
  \paragraph{Exercice 1}
  Développer et réduire les expressions suivantes~:
  \[
    \def\arraystretch{2}
    \begin{array}{c|c}
      A = 4x (-4x - 10) \quad&\quad B = 6(7x - 10)\\[3cm]
      \hline
      C = (2x + 1)(3x + 2) - 4x \quad&\quad D = 3x^2 + (x - 1)(x - 2)\\[3cm]
    \end{array}
  \]

  \paragraph{Exercice 2}
  Factoriser les expressions suivantes~:
  \[
    \def\arraystretch{2}
    \begin{array}{c|c}
      A = 12x - 12 \quad&\quad B = (2x+3)(x-5) + (x+2)(2x+3)\\[3cm]
    \end{array}
  \]

  \section*{Équations et inéquations}
  \paragraph{Exercice 1}
  Résoudre les équations suivantes~:
  \[
    \def\arraystretch{2}
    \begin{array}{c|c}
      x + 3 = 0 \quad&\quad 3x = 12\\[2cm]
      \hline
      \frac{2x}{5} = 4 \quad&\quad -x = 7\\[2cm]
      \hline
      \quad\quad x + 4 = 2x + 8 \quad&\quad 3x + 2 = x - 6 \quad\quad\\[2cm]
    \end{array}
  \]

  \paragraph{Exercice 2}
  Résoudre les inéquations suivantes en écrivant l'ensemble des solutions $S$~:
  \[
    \def\arraystretch{2}
    \begin{array}{c|c}
      2x + 4 < 6 \quad&\quad -4x + 3 \leq 15\\[2cm]
      \hline
      \quad\quad 4x + 3 \geq 2x + 3 \quad&\quad 8 > x - 4 \quad\quad\\[2cm]
    \end{array}
  \]

  \section*{Représentation graphique}
  \begin{center}
      \begin{tikzpicture}    
      % lignes horizontales
      \draw[dashed] (-4.7, -4) -- (+4.7, -4);
      \draw[dashed] (-4.7, -3) -- (+4.7, -3);
      \draw[dashed] (-4.7, -2) -- (+4.7, -2);
      \draw[dashed] (-4.7, -1) -- (+4.7, -1);
      \draw[dashed] (-4.7, +0) -- (+4.7, +0);
      \draw[dashed] (-4.7, +1) -- (+4.7, +1);
      \draw[dashed] (-4.7, +2) -- (+4.7, +2);
      \draw[dashed] (-4.7, +3) -- (+4.7, +3);
      \draw[dashed] (-4.7, +4) -- (+4.7, +4);
      
      % lignes verticales
      \draw[dashed] (-4, -4.7) -- (-4, +4.7);
      \draw[dashed] (-3, -4.7) -- (-3, +4.7);
      \draw[dashed] (-2, -4.7) -- (-2, +4.7);
      \draw[dashed] (-1, -4.7) -- (-1, +4.7);
      \draw[dashed] (+0, -4.7) -- (+0, +4.7);
      \draw[dashed] (+1, -4.7) -- (+1, +4.7);
      \draw[dashed] (+1, -4.7) -- (+1, +4.7);
      \draw[dashed] (+2, -4.7) -- (+2, +4.7);
      \draw[dashed] (+3, -4.7) -- (+3, +4.7);
      \draw[dashed] (+4, -4.7) -- (+4, +4.7);

      % axes
      \draw[->, thick] (+0, -5) -- (+0, +5);
      \draw[->, thick] (-5, +0) -- (+5, +0);

      % points
      \draw (0, 0) node[below left]{$0$};
      \fill(0, 0) circle (0.07cm);
      \draw (1, 0) node[below right]{$1$};
      \fill(1, 0) circle (0.07cm);
      \draw (0, 1) node[above left]{$1$};
      \fill(0, 1) circle (0.07cm);
      \draw (3, 3) node[above right]{$A$};
      \fill(3, 3) circle (0.07cm);
      \draw (1, -1) node[below left]{$B$};
      \fill(1, -1) circle (0.07cm);
      \draw (-2, 3) node[below left]{$C$};
      \fill(-2, 3) circle (0.07cm);
      \draw (3, -1) node[below right]{$D$};
      \fill(3, -1) circle (0.07cm);
    \end{tikzpicture}
  \end{center}

  \paragraph{1)}
  Placer dans le repère orthonormé les points suivants~:
  \[
    E(1~; 1) \quad\quad F(2~; 1) \quad\quad G(-1~; 3) \quad\quad H(-3~; -2)
  \]

  \paragraph{2)}
  Donne les coordonnées des points $A$, $B$, $C$ et $D$ dans le tableau ci-dessous~:
  \[
    \def\arraystretch{1.5}
    \begin{array}{c|c|c|c|c}
      & \quad A \quad & \quad B \quad & \quad C \quad & \quad D \quad\\
      \hline
      \mathrm{Abscisses} &&&&\\
      \hline
      \mathrm{Ordonnées} &&&&
    \end{array}
  \]

  \paragraph{3)}
  Tracer les doites $(A~;B)$ et $(C~;D)$ puis déterminer les coordonnées du point d'intersection $I$~:
  \[
    \mathrm{coordonnées\ de}\ I \ =\ \dots
  \]
\end{document}